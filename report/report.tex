\documentclass[a4paper,titlepage]{article}
\usepackage[english]{babel}
\usepackage{cite}
\usepackage[utf8]{inputenc}
\usepackage[hyphens]{url}
\usepackage[hidelinks]{hyperref}

\begin{document}
\begin{titlepage}
	\centering
	{\scshape\LARGE Utrecht University \par}
	\vspace{1cm}
	{\scshape\Large Experimentation Project \par}
	\vspace{1.5cm}
	{\huge\bfseries Experimental Analysis Of The Current State Of GPUs For Database Query Processing\par}
	\vspace{2cm}
	
	{\Large\itshape Patrick Kostjens \par}
	\href{mailto:p.a.r.kostjens@students.uu.nl}{p.a.r.kostjens@students.uu.nl}
	\vfill
	
	supervised by\par
	Drs.~Hans Philippi
	\vfill

% Bottom of the page
	{\large \today\par}
\end{titlepage}

\begin{abstract}

\textbf{Keywords:} query processing, graphics processing unit, relational database, parallel processing, CUDA
\end{abstract}

\section{Introduction}
Over the past years or even decades a lot of work has been done on database query processing. Most of this work has focused on query processing using CPUs. When developing algorithms for CPUs, algorithms are developed for a single or a few fast cores. %TODO citations

Another processor that is present in a computer is the Graphics Processing Unit (GPU). A GPU has a very different architecture. It has a lot of cores (hundreds or even thousands), but each core is a lot slower than a single CPU core. Originally, GPUs were mostly used for, for example, games or the rendering of images or videos, but in 2002, Thompson et al. \cite{thompson2002} published a paper on using GPUs for general purpose computing.

From that point forward, a lot of work has been done on database query processing using GPUs as well. We will discuss this work in some more detail in section \ref{sec:related-work}. Some impressive results are in this previous work. However, a lot still has to be done as well. First of all, most of the previous work focuses on a specific aspect of query processing, like joining or filtering. While results in such a specific aspect are definitely useful, those results need to be combined in a single implementation to see how a GPU would perform when processing an actual query instead of only a part of it.

An extra challenge in this area is the fact that the comparison between CPU query processing and GPU query processing may have different results in different years because CPUs and GPUs may be developed at different speeds. This means that initially it may be more efficient to perform a certain operation on a CPU because the GPU version of the algorithm is slower, while a couple of years later this might be the other way around when using the same algorithms.

Last year I also created a general overview of the possibilities \cite{kostjens2015}. However, in this experimental analysis of the current state of database query processing on GPUs, I will try to implement and combine some query operators on GPUs and CPUs to compare them to each other. This way, I hope to find how useful GPUs currently are compared to CPUs for database query processing. I will also try to make a comparison with the results found in the original work and I will try to combine the operations to find how useful the GPU is for processing more complete queries instead of only parts of them. This work will therefore be more detailed than my overview from last year and it will contain recent results for the discussed algorithms.

I will first discuss some related work in section \ref{sec:related-work}. Some of this work is also used to implement the GPU operations. Next, we will look at an overview of the current possibilities and some choices that were made in section \ref{sec:overview}. After that, we will discuss the implemented algorithms in section \ref{sec:implementation}. We will then look at the results in section \ref{sec:results}. Next, we will have a discussion about the results and the general state of GPUs for query processing in section \ref{sec:discussion}. Finally, we will conclude with some final remarks and a small summary in section \ref{sec:conclusion}.

\section{Related work}
\label{sec:related-work}
Since the paper by Thompson et al. in 2002 on general purpose computing using GPUs \cite{thompson2002}, a lot of research has been done on the topic of general purpose GPU's. One of the more specific topics in this area is that of database query processing using GPUs. Quite some research has been done in this area as well \cite{bakkum2010}, \cite{fang2007}, \cite{kaldeway2010}. 

So far, a lot of papers have focused on specific aspects of query processing. For example, He et al. \cite{he2008} and Kaldeway et al. \cite{kaldeway2010} researched joins, while, for example, Bakkum et al. \cite{bakkum2010} and Govinderaju et al. \cite{govindaraju2004} looked at algebraic selection (filtering).  Fang et al. \cite{fang2007} even did some work on combining the processing power of CPUs and GPUs.

Although quite a bit of work has been done on query processing using GPUs, a lot more research has been done on query processing using CPUs. This work also started a lot earlier. For example, in 1984, Bratbergsengen published a paper on the implementation of several relational algebra operations \cite{bratbergsengen1984}. These algebra operations are fundamental parts of database query processing.

\section{Overview}
\label{sec:overview}
Before being able to start the implementation of any actual GPU algorithms, a few choices had to be made. For example, which GPU to use. Other important choices include the development environment, and source of test data. Since I already had a capable, moderately recent GPU available I chose to use that one. This GPU is a GTX 660 Ti from Nvidia \cite{gtx660ti}, which was released in August 2012. This GPU is accompanied by Intel's i5-750 CPU from late 2009 \cite{i5-750}.

As the development environment, Visual Studio 2013 is used to work with CUDA \cite{CUDA} and C++. CUDA is NVIDIA's own parallel computing platform that I will be using to utilize the GPU. The well known TPC-H benchmark will be used as the source for test data \cite{tpc-h}.

Some differences between CPUs and GPUs exist that are very important for the development of query processing algorithms for GPUs. First of all, as mentioned already, GPUs have a lot of relatively slow cores, while CPUs have a few fast cores. This makes the nature of the algorithms for both very different since GPUs can only work with highly parallel algorithms.

Another important difference is related to the parallelism. While atomic operations for GPUs do exist, using them can significantly slow down your algorithm. If, for example, every core needs to get a memory address to write its results to, you could get an address and increase it atomically on a CPU. However, on a GPU, when hundreds or even thousands of cores are doing these atomic operations on the same memory address, this can cause significant waiting times, thereby slowing down the algorithm. This means alternative solutions for this problem, which we will discuss later, will have to be developed.

These two differences are the most important for the design of algorithms for GPUs. A more extensive overview of the differences between CPUs and GPUs can be found in my overview from last year \cite{kostjens2015}.

\section{Implementation}
\label{sec:implementation}

\section{Results}
\label{sec:results}

\section{Discussion}
\label{sec:discussion}

\section{Conclusion}
\label{sec:conclusion}

\bibliography{references}{}
\bibliographystyle{acm}

\end{document}
